\documentclass{article}
\usepackage[pdftex]{graphicx}
\usepackage{amsmath}
\usepackage{pgfplots}
\usetikzlibrary{shapes.geometric, arrows}
\pgfplotsset{width=10cm, compat=1.18}

\begin{document}
% --- Abstract ---
Signal processing is a foundational discipline in modern engineering, enabling the extraction, transformation, and interpretation of information from raw data. This document presents a comprehensive overview of advanced signal processing techniques, with a focus on the Dynamic Adaptive Wavelet Filter (DAWF) and its applications in biomedical and communications engineering. Key contributions include a new adaptive thresholding method, robust testing protocols, and a comparative analysis with state-of-the-art filters. The DAWF is benchmarked against both classical and modern approaches, demonstrating superior performance in non-stationary noise environments. The results are validated using both synthetic and real-world datasets, including EEG and radar signals.

% --- Introduction ---
The field of signal processing has evolved rapidly, driven by the increasing complexity of real-world signals and the demand for higher fidelity in data analysis. Traditional methods such as Fourier and Z-Transforms provide powerful tools for frequency analysis, but often fall short in non-stationary environments. Recent advances, including wavelet-based methods and adaptive filtering, have opened new avenues for research and application. This document aims to bridge the gap between theory and practice, providing both a rigorous mathematical foundation and practical implementation details for the DAWF framework. The motivation for this work stems from the need to process signals in challenging environments, such as deep-space communication, biomedical monitoring, and seismic data analysis. Figure~\ref{fig:intro_fig} shows a typical signal processing pipeline.
\begin{figure}[h!]
\centering
\includegraphics[width=0.7\textwidth]{intro_pipeline.jpg}
\caption{A generic signal processing pipeline from acquisition to analysis.}
\label{fig:intro_fig}
\end{figure}

% --- Theoretical Foundations ---
A discrete-time signal $x[n]$ is a sequence of real or complex numbers indexed by $n$. The Discrete-Time Fourier Transform (DTFT) is defined as:
\begin{equation}
X(e^{j\omega}) = \sum_{n=-\infty}^{\infty} x[n] e^{-j\omega n}
\end{equation}
The Z-Transform generalizes the DTFT:
\begin{equation}
X(z) = \sum_{n=-\infty}^{\infty} x[n] z^{-n}
\end{equation}
Stochastic processes, stationarity, and power spectral density (PSD) are also central concepts. Table~\ref{tab:theory} summarizes key properties.
\begin{table}[h!]
\centering
\caption{Key Theoretical Properties}
\begin{tabular}{|l|l|}
\hline
Property & Description \\
\hline
Stationarity & Invariance to time shifts \\
PSD & Power distribution in frequency \\
ROC & Region of convergence for Z-Transform \\
\hline
\end{tabular}
\label{tab:theory}
\end{table}

Random processes are often modeled as Gaussian or Poisson, with wide-sense stationarity (WSS) being a practical assumption. The autocorrelation function $R_x[m]$ and the power spectral density $S_x(e^{j\omega})$ are related by the Wiener-Khinchin theorem:
\begin{equation}
S_x(e^{j\omega}) = \sum_{m=-\infty}^{\infty} R_x[m] e^{-j\omega m}
\end{equation}
Figure~\ref{fig:theory_fig} shows a simulated spectrum.
\begin{figure}[h!]
\centering
\includegraphics[width=0.6\textwidth]{spectrum_example.jpg}
\caption{Simulated power spectral density of a WSS process.}
\label{fig:theory_fig}
\end{figure}

% --- The Proposed Framework: DAWF ---
The Dynamic Adaptive Wavelet Filter (DAWF) integrates adaptive filtering with multi-resolution wavelet analysis. Its architecture includes wavelet decomposition, adaptive thresholding, and signal reconstruction. Figure~\ref{fig:dawf_arch} illustrates the system.
\begin{figure}[h!]
\centering
\includegraphics[width=0.7\textwidth]{hardware_setup.jpg}
\caption{Block diagram of the DAWF system.}
\label{fig:dawf_arch}
\end{figure}
The update rule for the adaptive threshold is:
\begin{equation}
T_{k+1} = T_k - \mu \nabla J(T_k) + \alpha(T_k - T_{k-1})
\end{equation}
The DAWF is designed to adapt in real time, making it suitable for applications such as EEG denoising, radar signal enhancement, and wireless communications. Table~\ref{tab:dawf_params} lists key parameters.
\begin{table}[h!]
\centering
\caption{DAWF Parameters}
\begin{tabular}{|l|c|c|}
\hline
Parameter & Symbol & Typical Value \\
\hline
Adaptation Rate & $\mu$ & 0.01 \\
Momentum & $\alpha$ & 0.9 \\
Window Size & $N$ & 256 \\
\hline
\end{tabular}
\label{tab:dawf_params}
\end{table}

% --- Input Data and Database ---
The DAWF framework was validated using both synthetic and real-world datasets. Synthetic data included chirp and sine signals with added Gaussian and burst noise. Real-world data was sourced from EEG recordings. Table~\ref{tab:input_data} shows a sample of the input data.
\begin{table}[h!]
\centering
\caption{Sample Input Data}
\begin{tabular}{|c|c|c|}
\hline
Time (s) & Signal & Noise \\
\hline
0.01 & 0.25 & 0.05 \\
0.02 & 0.30 & -0.02 \\
0.03 & 0.45 & 0.10 \\
0.04 & 0.60 & 0.12 \\
0.05 & 0.55 & -0.08 \\
\hline
\end{tabular}
\label{tab:input_data}
\end{table}
The database schema is shown in Figure~\ref{fig:db_schema}.
\begin{figure}[h!]
\centering
\includegraphics[width=0.5\textwidth]{system_flowchart.jpg}
\caption{Database schema for DAWF results.}
\label{fig:db_schema}
\end{figure}

% --- Implementation Details ---
The DAWF was implemented in Python 3.8 using NumPy, SciPy, and PyWavelets. The codebase is modular, with separate files for data loading, core processing, and evaluation. The main algorithm is encapsulated in a DAWF class. Pseudocode for the main loop:
\begin{verbatim}
for window in signal.windows():
    coeffs = wavelet_decompose(window)
    snr = estimate_snr(coeffs)
    threshold = update_threshold(threshold, snr)
    coeffs_thr = apply_threshold(coeffs, threshold)
    rec = wavelet_reconstruct(coeffs_thr)
    filtered_signal.append(rec)
\end{verbatim}
A sample configuration file is shown below:
\begin{verbatim}
[DAWF]
adaptation_rate = 0.01
momentum = 0.9
window_size = 256
wavelet = db4
\end{verbatim}
Figure~\ref{fig:impl_fig} shows the software architecture.
\begin{figure}[h!]
\centering
\includegraphics[width=0.6\textwidth]{software_arch.jpg}
\caption{Software architecture of the DAWF implementation.}
\label{fig:impl_fig}
\end{figure}

% --- Testing and Verification ---
Testing included unit tests, integration tests, and real-world validation. Table~\ref{tab:unit_tests} lists unit test cases.
\begin{table}[h!]
\centering
\caption{Unit Test Cases}
\begin{tabular}{|l|l|l|}
\hline
Test ID & Description & Expected Outcome \\
\hline
UT-001 & Zero input & Output all zeros \\
UT-002 & High SNR & Threshold decreases \\
UT-003 & Low SNR & Threshold increases \\
UT-004 & Impulse input & Impulse response decays \\
\hline
\end{tabular}
\label{tab:unit_tests}
\end{table}
Integration tests used synthetic signals with known properties. Figure~\ref{fig:test_results} shows a test result.
\begin{figure}[h!]
\centering
\includegraphics[width=0.6\textwidth]{test_result.jpg}
\caption{Comparison of original, noisy, and filtered signals.}
\label{fig:test_results}
\end{figure}
A histogram of error values is shown in Figure~\ref{fig:error_hist}.
\begin{figure}[h!]
\centering
\includegraphics[width=0.5\textwidth]{error_hist.jpg}
\caption{Histogram of error values after filtering.}
\label{fig:error_hist}
\end{figure}

% --- Experimental Results and Discussion ---
The DAWF outperformed LMS and Savitzky-Golay filters in both synthetic and real-world tests. Figure~\ref{fig:results} shows a bar chart of PSNR values. Table~\ref{tab:results} summarizes results.
\begin{figure}[h!]
\centering
\includegraphics[width=0.6\textwidth]{results_chart.jpg}
\caption{PSNR comparison for different filters.}
\label{fig:results}
\end{figure}
\begin{table}[h!]
\centering
\caption{Performance Metrics}
\begin{tabular}{|l|c|c|}
\hline
Filter & PSNR (dB) & SDI \\
\hline
DAWF & 32.5 & 0.12 \\
LMS & 27.1 & 0.21 \\
Savitzky-Golay & 28.3 & 0.18 \\
\hline
\end{tabular}
\label{tab:results}
\end{table}
A scatter plot of SDI vs. MSE is shown in Figure~\ref{fig:scatter}.
\begin{figure}[h!]
\centering
\includegraphics[width=0.5\textwidth]{scatter_plot.jpg}
\caption{Scatter plot of SDI vs. MSE for different filters.}
\label{fig:scatter}
\end{figure}

% --- Conclusion ---
In summary, the DAWF framework provides a robust, adaptive solution for signal denoising in challenging environments. Its dynamic thresholding and multi-resolution analysis enable superior performance compared to traditional methods. Future work will focus on real-time implementation and extension to image and video signals. The results of this study have implications for a wide range of applications, from medical diagnostics to wireless sensor networks. Figure~\ref{fig:future} illustrates a potential future application.
\begin{figure}[h!]
\centering
\includegraphics[width=0.5\textwidth]{future_app.jpg}
\caption{Conceptual illustration of DAWF in a future IoT sensor network.}
\label{fig:future}
\end{figure}

% --- References ---
\begin{thebibliography}{9}
\bibitem{oppenheim}
A. V. Oppenheim and R. W. Schafer, "Discrete-Time Signal Processing," Prentice Hall, 2009.
\bibitem{haykin}
S. Haykin, "Adaptive Filter Theory," Prentice Hall, 2013.
\bibitem{mallat}
S. Mallat, "A Wavelet Tour of Signal Processing," Academic Press, 1999.
\bibitem{proakis}
J. G. Proakis and D. G. Manolakis, "Digital Signal Processing: Principles, Algorithms, and Applications," Pearson, 2007.
\bibitem{kailath}
T. Kailath, A. H. Sayed, and B. Hassibi, "Linear Estimation," Prentice Hall, 2000.
\bibitem{vetterli}
M. Vetterli and J. Kovačević, "Wavelets and Subband Coding," Prentice Hall, 1995.
\end{thebibliography}

\end{document}
