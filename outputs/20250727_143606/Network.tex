The steps to run the network are as follows:

\begin{enumerate}
\item New transactions are broadcast to all nodes.
\item Each node collects new transactions into a block.
\item Each node works on finding a difficult proof-of-work for its block.
\item When a node finds a proof-of-work, it broadcasts the block to all nodes.
\item Nodes accept the block only if all transactions in it are valid and not already spent.
\item Nodes express their acceptance of the block by working on creating the next block in the chain, using the hash of the accepted block as the previous hash.
\end{enumerate}

enumerate

Nodes always consider the longest chain to be the correct one and will keep working on extending it. If two nodes broadcast different versions of the next block simultaneously, some nodes may receive one or the other first. In that case, they work on the first one they received, but save the other branch in case it becomes longer. The tie will be broken when the next proof-of-work is found and one branch becomes longer; the nodes that were working on the other branch will then switch to the longer one.

New transaction broadcasts do not necessarily need to reach all nodes. As long as they reach many nodes, they will get into a block before long. Block broadcasts are also tolerant of dropped messages. If a node does not receive a block, it will request it when it receives the next block and realizes it's missing one.