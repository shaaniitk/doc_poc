\section{System Summary and Benefits}

We have proposed a system for electronic transactions without relying on trust. The system is based on the following key components and benefits:

\begin{itemize}
    \item \textbf{Digital Signatures}: Coins are made from digital signatures, providing strong control of ownership.
    \item \textbf{Peer-to-Peer Network}: A decentralized network is used to record a public history of transactions.
    \item \textbf{Proof-of-Work}: Transactions are recorded using proof-of-work, making it computationally impractical for an attacker to change the transaction history if honest nodes control a majority of CPU power.
    \item \textbf{Robustness and Simplicity}: The network is robust due to its unstructured simplicity. Nodes work independently with little coordination.
    \item \textbf{Anonymity}: Nodes do not need to be identified, as messages are not routed to any particular place and only need to be delivered on a best-effort basis.
    \item \textbf{Flexibility}: Nodes can leave and rejoin the network at will, accepting the proof-of-work chain as proof of what happened while they were gone.
    \item \textbf{Consensus Mechanism}: Nodes vote with their CPU power, expressing their acceptance of valid blocks by working on extending them and rejecting invalid blocks by refusing to work on them. This consensus mechanism enforces any needed rules and incentives.
\end{itemize}

The proposed system ensures that transactions are secure, transparent, and tamper-proof, providing a robust framework for electronic transactions without the need for a trusted third party.