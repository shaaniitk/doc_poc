Once the latest transaction in a coin is buried under enough blocks, the spent transactions before it can be discarded to save disk space. To facilitate this without breaking the block's hash, transactions are hashed in a Merkle Tree, with only the root included in the block's hash. Old blocks can then be compacted by stubbing off branches of the tree. The interior hashes do not need to be stored.

A block header with no transactions would be about 80 bytes. If we suppose blocks are generated every 10 minutes, 80 bytes * 6 * 24 * 365 = 4.2MB per year. With computer systems typically sold with 2GB of RAM as of 2008, and Moore's Law predicting current growth of 1.2GB per year, storage should not be a problem even if the block headers must be kept in memory.