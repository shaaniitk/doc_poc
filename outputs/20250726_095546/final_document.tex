\documentclass{article}
\usepackage[pdftex]{graphicx}
\usepackage{amsmath}
\usepackage{pgfplots}
\usetikzlibrary{shapes.geometric, arrows}
\pgfplotsset{width=10cm, compat=1.18}

\begin{document}

% Abstract
\documentclass{article}
\begin{document}
\begin{abstract}
We propose a peer-to-peer electronic cash system that enables direct online payments between parties without financial institutions.

Digital signatures address part of the solution, but the main benefits are lost if a trusted third party is still required to prevent double-spending.

Our solution to the double-spending problem uses a peer-to-peer network that timestamps transactions by hashing them into an ongoing chain of hash-based proof-of-work, forming an unalterable record.

The longest chain serves as proof of the sequence of events and demonstrates it originated from the largest pool of CPU power.

As long as a majority of CPU power is controlled by honest nodes, they will generate the longest chain and outpace attackers.

The network requires minimal structure, with messages broadcast on a best-effort basis, and nodes can freely join and leave, accepting the longest proof-of-work chain as proof of events during their absence.
\end{abstract}
\end{document}

% 1. Introduction
\documentclass{article}
\begin{document}
\begin{abstract}
We propose a peer-to-peer electronic cash system that enables direct online payments between parties without financial institutions.

Digital
\end{abstract}

\section*{Introduction}

Commerce on the Internet has come to rely almost exclusively on financial institutions serving as trusted third parties to process electronic payments. While the system works well enough for most transactions, it still suffers from the inherent weaknesses of the trust-based model.

Completely non-reversible transactions are not really possible, since financial institutions cannot avoid mediating disputes. The cost of mediation increases transaction costs, limiting the minimum practical transaction size and cutting off the possibility for small casual transactions. This also results in the broader loss of the ability to make non-reversible payments for non-reversible services.

With the possibility of reversal, the need for trust spreads. Merchants must be wary of their customers, hassling them for more information than they would otherwise need. A certain percentage of fraud is accepted as unavoidable.

These costs and payment uncertainties can be avoided in person by using physical currency, but no mechanism exists to make payments over a communications channel without a trusted party.

What is needed is an electronic payment system based on cryptographic proof instead of trust, allowing any two willing parties to transact directly with each other without the need for a trusted third party. Transactions that are computationally impractical to reverse would protect sellers from fraud, and routine escrow mechanisms could easily be implemented to protect buyers.

In this paper, we propose a solution to the double-spending problem using a peer-to-peer distributed timestamp server to generate computational proof of the chronological order of transactions. The system is secure as long as honest nodes collectively control more CPU power than any cooperating group of attacker nodes.

\end{document}

\end{document}