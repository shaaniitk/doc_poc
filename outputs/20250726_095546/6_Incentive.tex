\documentclass{article}
\begin{document}
\section{Incentive Mechanism}

By convention, the first transaction in a block is a special transaction that starts a new coin owned by the creator of the block.

This adds an incentive for nodes to support the network, and provides a way to initially distribute coins into circulation, since there is no central authority to issue them.

The steady addition of a constant of amount of new coins is analogous to gold miners expending resources to add gold to circulation.

In our case, it is CPU time and electricity that is expended.

The incentive can also be funded with transaction fees.

If the output value of a transaction is less than its input value, the difference is a transaction fee that is added to the incentive value of the block containing the transaction.

Once a predetermined number of coins have entered circulation, the incentive can transition entirely to transaction fees and be completely inflation free.

The incentive may help encourage nodes to stay honest. If a greedy attacker is able to assemble more CPU power than all the honest nodes, he would have to choose between using it to defraud people by stealing back his payments, or using it to generate new coins. He ought to find it more profitable to play by the rules, such rules that favour him with more new coins than everyone else combined, than to undermine the system and the validity of his own wealth.
\end{document}