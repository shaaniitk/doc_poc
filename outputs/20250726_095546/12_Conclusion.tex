\documentclass{article}
\begin{document}

\section*{System Summary and Benefits}

We propose a system for electronic transactions without relying on trust, using the following framework:

\begin{itemize}
    \item \textbf{Digital Signatures}: Coins are made from digital signatures, providing strong control of ownership.
    \item \textbf{Peer-to-Peer Network}: A network using proof-of-work to record a public history of transactions.
    \item \textbf{Double-Spending Prevention}: The proof-of-work chain becomes computationally impractical for an attacker to change if honest nodes control a majority of CPU power.
    \item \textbf{Robust and Unstructured}: Nodes work independently with little coordination, and messages are delivered on a best-effort basis.
    \item \textbf{Consensus Mechanism}: Nodes vote with their CPU power, accepting valid blocks by extending them and rejecting invalid blocks by refusing to work on them.
\end{itemize}

\subsection*{Benefits}

\begin{itemize}
    \item \textbf{Decentralization}: No need for a central authority or identification of nodes.
    \item \textbf{Robustness}: The network is robust due to its unstructured simplicity and consensus mechanism.
    \item \textbf{Flexibility}: Nodes can leave and rejoin the network at will, accepting the proof-of-work chain as proof of what happened while they were gone.
    \item \textbf{Security}: The proof-of-work system makes it computationally impractical for an attacker to change the transaction history.
\end{itemize}

\end{document}