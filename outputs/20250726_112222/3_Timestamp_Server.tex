### Timestamp Server

The solution we propose begins with a timestamp server. A timestamp server works by taking a hash of a block of items to be timestamped and widely publishing the hash, such as in a newspaper or Usenet post. The timestamp proves that the data must have existed at the time, obviously, in order to get into the hash. Each timestamp includes the previous timestamp in its hash, forming a chain, with each additional timestamp reinforcing the ones before it.

This chain of hashes forms a proof-of-work, as described by Haber and Stornetta. To implement a timestamp server on a peer-to-peer basis, we will need to use a proof-of-work system similar to Adam Back's Hashcash, rather than newspaper or Usenet publications. The proof-of-work involves scanning for a value that when hashed, such as with SHA-256, the hash begins with a number of zero bits. The average work required is exponential in the number of zero bits required and can be verified by executing a single hash.

For our timestamp network, we implement the proof-of-work by incrementing a nonce in the block until a value is found that gives the block's hash the required number of leading zero bits. Once the latest block has been hashed, it cannot be changed without redoing the work. As later blocks are chained after it, the work to change the block would include redoing the work for each subsequent block.

To compensate for increasing hardware speed and varying interest in running nodes over time, the proof-of-work difficulty is determined by a moving average targeting an average number of blocks per hour. If they're generated too fast, the difficulty increases.