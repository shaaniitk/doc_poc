\documentclass{article}
\usepackage[utf8]{inputenc}
\usepackage{amsmath}

\title{Blockchain Security Analysis}
\author{Security Research Team}
\date{}

\begin{document}

\maketitle

% --- Abstract ---
This paper analyzes the security implications of blockchain technology, focusing on consensus mechanisms and attack vectors. We examine various blockchain implementations and their resistance to different types of attacks.

% --- Introduction ---
Blockchain technology has emerged as a revolutionary approach to distributed computing and digital asset management. The security of blockchain systems depends on several key factors including consensus mechanisms, cryptographic primitives, and network topology.

% --- Security Models ---
We define several security models for blockchain systems:

1. **Byzantine Fault Tolerance**: The system can tolerate up to f faulty nodes out of 3f+1 total nodes.

2. **Proof-of-Work Security**: Security is based on computational power majority.

3. **Proof-of-Stake Security**: Security is based on economic stake majority.

% --- Attack Vectors ---
Common attack vectors include:

- **51% Attack**: Controlling majority of network power
- **Double Spending**: Spending the same coins twice
- **Sybil Attack**: Creating multiple fake identities
- **Eclipse Attack**: Isolating nodes from the network

% --- Consensus Analysis ---
Different consensus mechanisms provide different security guarantees:

\begin{equation}
P_{attack} = \left(\frac{q}{p}\right)^z
\end{equation}

Where $P_{attack}$ is the probability of successful attack, $q$ is attacker power, $p$ is honest power, and $z$ is confirmation depth.

% --- Conclusion ---
Blockchain security requires careful consideration of consensus mechanisms, network assumptions, and economic incentives. No single approach provides perfect security, but proper design can achieve acceptable risk levels for most applications.

\end{document}